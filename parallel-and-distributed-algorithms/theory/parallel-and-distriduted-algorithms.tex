\documentclass[a4paper]{article}

\usepackage[T1]{fontenc}
\usepackage[utf8]{inputenc}
\usepackage[italian]{babel}
\usepackage{amssymb}
\usepackage{hyperref}
\usepackage{mathtools}
\usepackage{amsthm}
\usepackage[ruled,vlined,noend]{algorithm2e}

% ===== Math =====
\usepackage{nccmath}
% ===== Graph =====
\usepackage{tikz}
% ===== Multicol =====
\usepackage{blindtext}
\usepackage{multicol}
% ===== Cancel =====
\usepackage{cancel}
% ===== Code =====
\usepackage{listings} 
\lstdefinestyle{mystyle}{
    breakatwhitespace=false,                   
    captionpos=b,                    
    keepspaces=true,                 
    numbers=left,                    
    numbersep=5pt,                  
    showspaces=false,                
    showstringspaces=false,
    showtabs=False,                  
    tabsize=2
}

\lstset{style=mystyle}

\usepackage{setspace}
\singlespacing
% ===================
\mathtoolsset{showonlyrefs}  
\hypersetup{
    colorlinks=true,
    linkcolor=black,
    filecolor=black,      
    urlcolor=black,
}

\AtBeginDocument{\renewcommand\proofname{Proof}}

\newcommand{\pluseq}{\mathrel{{+}{=}}}

\newtheorem{theorem}{Theorem}
\newtheorem{corollary}{Corollary}
\newtheorem{lemma}{Lemma}
\newtheorem{remark}{Remark}
\newtheorem{definition}{Definition}

\setcounter{secnumdepth}{3}
\setcounter{tocdepth}{3}

\title{Parallel and Distributed Algorithms.tex}
\author{Federico Bruzzone}
\makeindex
\begin{document}
\maketitle
\newpage
% \setlength{\parskip}{0.15em}
\tableofcontents
\setlength{\parindent}{0pt}
\setlength{\parskip}{0.8em}
\newpage


\section{Introduction}

\textbf{Problem of sequential algorithm:}

\begin{itemize}
 \item design: techniques such as divide et impera, dyniamic programming, greedy, ...
 \item evaluate performance: time complexity, space complexity
 \item develop: implement with a programming language
\end{itemize}

Similarly, the same problem can be faced for Parallel and Distributed Algorithms (PDAs).

\textbf{PDAs are solved by a pool of executors, and there are two cases:}

\begin{enumerate}
 \item pool of executors such that:\\
 - have a central clock\\
 - can be share resources (i.e., memory)\\
 \begin{multicols}{2}
 [For example: Sum of 4 numbers]
  \begin{tikzpicture}[node distance={15mm}, thick, main/.style = {draw, circle}] 
   \node[main] (A)              {$1/A$}; 
   \node[main] (B) [right of=A] {$2/B$}; 
   \node[main] (C) [below of=A] {$3/C$}; 
   \node[main] (D) [below of=B] {$4/D$}; 
		
   \draw (A) -- (C);
   \draw (B) -- (D);
   \draw[->] (A) -- (B);
   \draw[->] (C) -- (D);
   \end{tikzpicture} 
		
   Number: A, B, C, D\\
   Processor: 1, 2, 3, 4\\
    \begin{algorithm}[H]
    	 \SetAlgoLined
    	 \KwIn{$G=(V,E)$}
    	 \KwResult{Sum of $V$}
     send($1$, $2$) $\rightarrow$ $A+B$\\
     send($3$, $3$) $\rightarrow$ $C+D$\\
     send($2$, $4$) $\rightarrow$ $A+B+C+D$
     \caption{Sum of 4}
	 \end{algorithm}
	 4 clock cycles
	\end{multicols}
 \item pool of executors such that:\\
 - each has its own central clock\\
 - they are connected to each outher (we cannot assume that the memory is shared)
 \begin{multicols}{2}
 [For example: Sum of 4 numbers]
  \begin{tikzpicture}[node distance={15mm}, thick, main/.style = {draw, circle}] 
   \node[main] (A)              {$1/A$}; 
   \node[main] (B) [right of=A] {$2/B$}; 
   \node[main] (C) [below of=A] {$3/C$}; 
   \node[main] (D) [below of=B] {$4/D$}; 
		
   \draw (A) -- (C);
   \draw (B) -- (D);
   \draw[->] (A) -- (B);
   \draw[->] (C) -- (D);
  \end{tikzpicture} 
		
  Number: A, B, C, D\\
  Processor: 1, 2, 3, 4\\
   \begin{algorithm}[H]
    	\SetAlgoLined
    \KwIn{$G=(V,E)$}
    \KwResult{Sum of $V$}
    send($1$, $2$) \& send($3$, $3$)\\
    // may come at different times\\
    // the processor might wait to do the final sum\\
    // could be usefull coordination signals
    \caption{Sum of 4}
   \end{algorithm}
 \end{multicols}
\end{enumerate}

In parallel algorithms the main factor is \textbf{Time}\\
- To sum 4 number the cycles are 4, but to sum 1000 number the cycles are 10\\
In distributed algorithms the main facotor is \textbf{coordination}\\
- less messages = faster

\subsection{Course Program}

For each paradigm:

\begin{enumerate}
 \item theortical model
 \item evaluate the performance
 \item simple problem to learn the techniques
\end{enumerate}

\textbf{Parallel case}

\begin{enumerate}
 \item A:
  \begin{enumerate}
   \item PRAM (immediate communication) $\rightarrow$ shared memory
   \item computing resources $\rightarrow$ time, hardware=\#processor
   \item problems:\\
    - summation\\
    - prefixed sums\\
    - sorting\\
  \end{enumerate}
 \item B:
  \begin{enumerate}
   \item shared memory model\\   
    - Linear array\\ 
    - Mesh
   \item computing resources
   \item problems:\\ 
    - Shuffle
    - Max
    - Sorting
  \end{enumerate}
\end{enumerate}

\textbf{Distributed case}

\begin{enumerate}
 \item definition of abstract  model
 \item computing resource $\rightarrow$ time, \#messages=congestion
 \item problems:\\
 - broadcast\\
 - wake up\\
 - traversal\\
 - spanning tree\\
 - election\\
 - routing
\end{enumerate}
\section{Time}

In both cases, parallel algorithm and distributed algotithm the crucial resource is time.\\

Formal definition:\\

$T(x) = $ number of elementary operations on $x$ ($x$ is the instance)\\
$t(n) = max\{T(x) \ | \ x \in \sum_{}^{n} \} \ | \  n \ is \ the \ length \ of \ the \ input$

\begin{remark}
Often we will not be interested in a precise evaulation of $T(n)$, but in its growth rate. We will use $O \ \Omega \ \Theta \ $
\end{remark}

\begin{definition}
Be $f,g : \mathbb{N} \rightarrow \mathbb{N}$ two natural function. We can say that:\\
- $f(n) = O(g(n))$ iff $\exists$ a constant $c > 0$ and $n_0 \in \mathbb{N} \ | \ f(n) \leq c*g(n) \ \forall \ n \geq n_0$\\
- $f(n) = \Omega(g(n))$ iff $\exists$ a constant $c > 0$ and $n_0 \in \mathbb{N} \ | \ f(n) \geq c*g(n) \ \forall \ n \geq n_0$\\
- $f(n) = \Theta(g(n))$ iff $\exists$ two constant $c1, c2 > 0$ and $n_0 \in \mathbb{N} \ | \ c1*g(n) \leq \ f(n) \ \leq c2*g(n) \ \forall \ n \geq n_0$\end{definition}

\begin{enumerate}
 \item $t(n)$ is evaluated in a particular model of computation
 \item the cost criterion must be chosen: uniform of logarithmic
\end{enumerate}

We count the primitives that our model of computation gives us.

\textbf{Palindrome}

\begin{algorithm}[H]
 \SetAlgoLined
 \KwIn{$x \in \{0,1\}^*$}
 \KwResult{$x$ is palindrome}
 \For{$i=0,j=|x|; \ i<j \ \& \ x_i=x_j; \ i++, j--$}{}
 \Return{$i \geq j$}
 \caption{Palindrome}
\end{algorithm}

This program:

\begin{itemize}
 \item on RAM has $t(n) == O(n)$
 \item on DTM (deterministic Turing machine) has $t(n) == O(n^2)$
\end{itemize}

\textbf{Factorial}

\begin{algorithm}[H]
 \SetAlgoLined
 \KwIn{$x \in \{0,1\}^*$}
 \KwResult{$n!$}
 $k \gets 1$\\
 \For{$i=1; \ i \leq x; \ k=k*i, i++$}{}
 \Return{$k$}
 \caption{Factorial}
\end{algorithm}

This program:\\

\begin{itemize}
 \item using multiplication as an elementary operation of RAM\\
$t(\lg{n}) = n$
 \item on DTM we should write the result in binary is $t(\lg{n}) \geq \lg{n^n} \ = \ n \lg{n}$
\end{itemize}

\textbf{Uniform cost criterion}\\
Elementary operations needed one unit time\\

\textbf{Logarithmic cost criterion}\\
Each elementary operations has a cost that depends on the number of bits of the operands

\textbf{Complexity theory}

$t(n)$ is:\\
\begin{itemize}
 \item logarithmic when is $O(\lg{n})$
 \item polylogarithmic when is $O(\lg^k{n})$
 \item linear when is $O(n)$
 \item polynomial when is $O(n^k) \ \forall k>0$ 
 \item exponential when is NOT $O(n^k) \ \forall k>0$ 
\end{itemize}

\textbf{Efficiency concept}

\begin{definition}
A problem is efficienty solved in time if and only if is solved in a DTM in polynomial time.
\end{definition}

$P$ = decision problem class solved in polynomial time\\
$FP$ = general problem class solved in polynomial time\\
$NP$ = decision problem class solved in polynomial time on nDTM (non-deterministic Turing machine)

\begin{remark}
 the asymptotic evaluation of times can hide constants that make the difference in practice\\
  i.e., sometimes we prefer quick-sort ($O(n^2)$) over merge-sort ($O(n\lg{n})$)\\
 the degree of the polynomial must be low\\
  i.e., $n^{1000}$ on small instance is worse than exponential algorithms\\
\end{remark}

\section{Parallel Algorithms}

\textbf{Problems}:

\begin{enumerate}
 \item \textbf{synthesis}:\\
  How to build algorithms? We can draw inspiration from sequential algoithms, but we need to have innovative ideas.
 \item \textbf{evaluation}:\\
 How to evaluate performance?\\
  - $time$\\
  - $hardware$ = \# processor\\
 What is efficient?\\
  - $E$ = efficiency\\
   It is usefull to have a ratio between $time$ and $hardware$
 \item \textbf{universality}: \\
  Am I able to define a class of problems for efficient parallel algorithm?\\
  Sequential = $FP$\\
  Parallel = $NC$\\
  $NC \rightarrow$ problem solved by faster parallel algorithm with:\\
   - polylogarithmic time\\
   - polynomial hardware
  \begin{proof}
  $NC \subseteq FP$\\
  To obtain a sequential algorithm starting from a parallel one, it is enough to simulate "with a processor" the work of the processors involved in sequence.
  \end{proof}
\end{enumerate}

\textbf{Types of parallel architecture}

\begin{enumerate}
 \item \textbf{shared memory}:\\ 
  - Invisible property:\\
   We have a central clock\\
  - Communication property:\\
   Constant time communication between processors:\\
   For example: \textit{If $P_j$ wants communicate $x$ to $P_i$:}\\
   - $P_j$ writes $x$ in central memory\\
   - $P_i$ read $x$ from central memory
 \item \textbf{distributed memory}:
  - Invisible property:\\
   We have a central clock\\
  - Communication property:\\
   The communication depends on the distance between processors: if $P_j$ wants communicate with $P_i$, we should ask ourselves how many processors there are in between.
\end{enumerate}

\subsection{PRAM model (Parallel RAM)}

It is composed by:
\begin{itemize}
 \item A central memory (central clock)
 \item $n$ sequential RAMs that have thier own private memory
\end{itemize}

Types of instruction:
\begin{itemize}
 \item logical arithmetic operations
 \item operations from/to central memory:\\
 - $STORE \ R[n] \ M[n]$\\
 - $LOAD \  R[n^1] \ M[n^1]$
\end{itemize}

We are able to manipulate only the data in private memory, doing this the communication come in $O(1)$

\textbf{From of instrucitions}

\begin{algorithm}[H]
 \SetAlgoLined
 \For{$i \ \in \ I$}{$instruction_i$}
 \caption{From of instrucitions}
\end{algorithm}

The processors with index in $I$ execute $instruction_i$\\
The processors with index not in $I$ execute $null$

What's \textbf{$instruction_i$}?\\
There are two types of different architecture:
\begin{itemize}
 \item SIMD: \textit{single instruction multiple data}
 \item MIMD: \textit{multiple instruction multiple data}
\end{itemize}

$instruction_i$ is for $i \neq j$ the same instruction

Depending on the ability to access memory $M$ (central) we also have different architectures:
\begin{itemize}
 \item EREW:\\
 - read/write in the same cell of $M$
 \item CREW:\\
 - simultaneous reading
 - no simultaneous writing
 \item CRCW:\\
 - simultaneous reading/writing\\
 For simultaneous writing we have different policies:
 \begin{itemize}
  \item common: the processor can write the same data, otherwise system shutdown
  \item random: $P_i$ is chosen randomly
  \item max/min: wins $P_i$ with max/min data
  \item priority: wins $P_i$ with max priority
 \end{itemize}
\end{itemize}

\textbf{Computing Resources}

- Sequencial: $t(n)$, $s(n)$\\
- Parallel: $p(n)$, $T(n, p(n))$

\textbf{Example of P-RAM algorithm}

\begin{algorithm}[H]
 - Assuming that array $A$ have distict value\\
 - \# \ of \ proccessors = n\\
 \SetAlgoLined
 \KwIn{$A, \ n, \ x$}
 \KwResult{$find x in A$}
 $index \gets -1$\\
 \For{$i=0; \ to \ n-1$} {
  $P_i$: \If{$A[i] = x$} {
   $index = i$
  } 
 }
 \Return{$index$}
 \caption{Find}
\end{algorithm}

Parallel time = constant

If $A$ repeating elements $\rightarrow$ CRCW

\textbf{Informal definition}

$P(n)$ = number of processors required on input of length $n$.

$T(n, P(n))$ = Time required by an input of length $n$ and $P(n)$ processors. (Worst case)

\textbf{Formal definition}

$t_i(n) = max\{(t_i)^{(j)}_{(n)} \ | \ 1 \leq j \leq P(n)\}$

$T(n, P(n)) = \sum_{i=1}^{k(n)}{t_i(n)}$

$T$ depends on $k(n)$

$T$ depends on the input length $[logarithmic / uniform \ cost)$

$T$ depends on $P(n)$
\textbf{Broadcast on P-RAM EREW}

\begin{algorithm}[H]
 \SetAlgoLined
 \KwIn{$x$}
 %\KwResult{$find x in A$}
 $A[0] \gets x$\\
 \For{$i=0 \ to \ \log{n}-1$} {
  \For{$j=2^{i} \ to \ 2^{i+1}-1$ /* parallelize */ } { 
   $A[j] = A[j-2^{i}]$ 
  }
 } 
 \caption{Broadcast}
\end{algorithm}

$t(n, n) = O(\log{n})$

\textbf{Use Broadcast on P-RAM}

\begin{algorithm}[H]
 \SetAlgoLined
 \KwIn{$A, \ n, \ x$}
 \KwResult{}
 $Broadcast(x)$\\
 $index \gets -1$\\
 \For{$i=0 \ to \ n-1$ /* parallelize */} {
  \If{$A[i] = x[i]$} { 
   $index = i$ 
  }
 } 
 \Return{index}
 \caption{Find}
\end{algorithm}

- $t(n, n) = O(\log{n})$\\
- if $A[i]$ are distinct $\rightarrow$ EREW\\
- else $\rightarrow$ ERCW\\
- ERCW can be transformed in EREW (increasing the time function)

\subsubsection{Efficiency}

A first comparison between the times\\
$T(n,1) 	\leftrightarrow T(n, P(n))$\\ \\
There are two cases:
\begin{itemize}
 \item $T(n, P(n)) = \Theta(T(n,1))$ NO!
 \item $T(n, P(n)) = o(T(n,1))$ YES!
\end{itemize}

\textbf{Speed-up}

$S(n, P(n)) = \frac{T(n,1)}{T(n, P(n))}$\\
For example, if $S = 4$ the parallel algorithm is 4 times faster than sequential one.\\
If we are in the case: $T(n, P(n)) = o(T(n,1)) \implies S(n, P(n)) \rightarrow +\infty$. We are not cosidering $P(n)$\\

Problem: \textbf{SAT} (where the length of the formula is linearly related to the number of variables involved)\\
$T(n,1) = 2^n$ for the moment ($ \in NP$)\\
$T(n, P(n)) = O(n)$ using a processor for each assignment\\
$S(n, P(n)) = \frac{2^n}{n} \rightarrow +\infty$ beacuse $P(n)=2^n$

\textbf{Efficiency}

$E(n, P(n)) = \frac{S(n, P(n))}{P(n)} = \frac{T(n,1)^*}{P(n) \cdot T(n, P(n))}$\\
* means the best sequencial time ($o$ lower bound)\\
$0 \leq E(n, P(n)) \leq 1$

Problem: \textbf{SAT}\\
$E(n, P(n)) \simeq \frac{1}{n} \rightarrow 0 $
\begin{remark}
 when $E \rightarrow 0$ we are using lots of processors, that they are not used for a long time. 
\end{remark}

We can see $E \leq 1$ in two distinct ways:

\begin{enumerate}
 \item through Parallel Alg. $\rightarrow$ Sequencial Alg.\\
 The transformation: I sequence the parallel steps\\
 Time: $T^{\sim}(n,1)$\\
 $T(n,1) \leq T^{\sim}(n,1) \leq P(n) \cdot t_1(n) + \dots + P(n) \cdot t_{kn}(n) = P(n) \cdot \sum$\\
 We have that $T(n, 1) \leq P(n) \cdot T(n, P(N))$ (*)
 \begin{remark}
  from (*) $\frac{T(n,1)}{P(n)} \leq T(n, P(n))$\\
  if $T(n,1)$ is the best sequential time. The best time of the parallel algorithm would be distributed equally to processors the sequencial work.
 \end{remark}
 \begin{remark}
  from (*) $\frac{T(n,1)}{P(n) \cdot T(n,P(n))} \leq 1 \ | \ where \ P(n) \cdot T(n,P(n)) = E(n,P(n)))$\\
  if $T(n,1)$ is the best sequential time. The best time of the parallel algorithm would be distributed equally to processors the sequencial work.
 \end{remark}
 $0 \leq E \leq 1$
 \begin{itemize}
  \item $E \rightarrow 0$ is no good
  Given that $T(n,P(n)) = o(T(n,1))$ means $P(n)$ grows too fast\\
  The best that we can have is $E \rightarrow K \leq 1$ where $K$ is a constant
 \end{itemize}  
 \item J. Wyllie (1979 PhD Thesis)\\
 If $E \rightarrow 0$ then to improve the algorithm try to reduce $P(n)$ without degrading time\\
 We can see the validity of this principle by reducing the number of processors from $P \rightarrow P/K$
 Time:\\
 $T(n,P/K) \leq \sum_{i=1}^{K(n)}K \cdot t_i(n) = K \cdot \sum_{i=1}^{K(n)} t_i(n) = K \cdot T(n)$\\
 we have that: $T(n,P/K) \leq K \cdot T(n)$\\
 E grow by decreasing the processors\\
 $E(n, P/K) \leftrightarrow E(n,P)$\\
 $E(n, P/K) = \frac{T(n,1)}{P/K \cdot T(n, P/K)} \geq \frac{T(n,1)}{P \cdot T(n, P)}  = E(n,P)$\\
 This formula show us that decreasing the processors improves efficiency.\\
 $1 = E(n,1) = E(n, P/P) \geq E(n, P/K) \geq E(n,P)$\\
 It is important to be careful to keep $T(n, P/K) \geq o(T(n,1))$ because $E(n,1) = 1, but T(n, P=1) = T(n,1) that is sequential.$         
\end{enumerate}


\subsection{Parallel Sum}

Motivations:
\begin{enumerate}
 \item \textbf{technique}: divide-and-conquer
 \item \textbf{pattern}: to solve other associative operations
 \item \textbf{module}: subproblem of other prolems
\end{enumerate}

\textbf{Sequencial sum}\\
\begin{algorithm}[H]
 \SetAlgoLined
 \KwIn{$M[1], M[2], ..., M[n]$}
 \KwResult{$M[n] = \sum_{i=1}^{n}M[i]$}
 \For{$i=1 \ to \ n-1$} {
  $M[n] = M[n] + M[i]$ 
 }
 %\Return{}
 \caption{Sequencial sum ($T(n,1) = n-1$)}
\end{algorithm}

To parallelize this algorithm there are two ways.
\begin{enumerate}
 \item one sum per processor\\
 The heigh of the tree is $n-1$ and the efficiency is $E = \frac{\cancel{n-1}}{\cancel{(n-1)} \cdot (n-1)} \rightarrow 0$
 \item associative sum\\
 The heigh of the tree is $\log{n}$. \\
 This is of course possibile because $((\alpha+\beta)+\gamma)+\delta = (\alpha+\beta)+(\gamma+\delta)$
\end{enumerate}

\textbf{Parallel sum} (using second way)\\
\begin{algorithm}[H]
 \SetAlgoLined
 Let $n = 2^t$ (items to add)
 \KwIn{$M[1], M[2], ..., M[n]$}
 \KwResult{$M[n] = \sum_{i=1}^{n}M[i]$}
 \For{$j=1 \ to \ \log{n}$} {
  \For{$K=1 \ to \ n/2^j$ /*parallelize*/} {
   $M[2^{j}K] = M[2^{j}K] + M[2^{j}K - 2^{j-1}]$ 
  } 
 }
 \Return{$M[n]$}
 \caption{Parallel sum}
\end{algorithm}

\textbf{Instruction}:\\
\textit{1 step}: $M[2K] = M[2K] + M[2K-1] \ || \ 1 \leq K \leq n/2$\\
\textit{2 step}: $M[4K] = M[4K] + M[4K-2] \ || \ 1 \leq K \leq n/4$\\
\textit{3 step}: $M[8K] = M[8K] + M[8K-4] \ || \ 1 \leq K \leq n/8$\\
...\\
\textit{$\log{n}$ step}: $M[n] = M[n] + M[n/2] \ || \ 1$\\

\textbf{Is it EREW?}\\
No, read and write simultaneously in the same cell of M.\\

Processors $a,b \rightarrow a \neq b$ \\
Cells involved by $a$ and $b$:\\
\begin{tikzpicture}[node distance={15mm}, thick, main/.style = {draw}] 
\node[main] (1) {$2^{j}a$}; 

\node[main] (2) [right of=1] {$2^{j}a - 2^{j-1}$}; 
\node[main] (3) [below of=1] {$2^{j}b$}; 
\node[main] (4) [right of=3] {$2^{j}b - 2^{j-1}$};

\draw (1) -- (3);
\draw (1) -- (4);
\draw (2) -- (3);
\draw (2) -- (4);

\draw (3) -- (1);
\draw (3) -- (2);s
\draw (4) -- (1);
\draw (4) -- (2);

\end{tikzpicture} 

\textbf{Comparisons}:\\
$2^{j}a \neq 2^{j}b$ yes, for $a \neq b$\\
$2^{j}a \neq 2^{j}b - 2^{j-1}$ $\overset{by \ contradiction}{\Rightarrow}$ $2a = 2b - 1 \Rightarrow a = \frac{2b-1}{2} \notin \mathbb{N}$\\
!!! EREW algorithm

\begin{proof}
\textbf{(*)} $M[2^{j}K] = M[2^{j}K] + \dots + M[2^{j}(K-1)+1]$\\
Per $j=log{n}$, obviously $K=1$ e we have that:\\
$M[n] = M[n] + \dots + M[1]$.\\
Proof by induction:\\
\textit{Base case:} $j=1$ and $1 \leq K \leq n/2$	\\
Algorithm instruction: $M[2K] = M[2K] + \dots + M[2K-1]$.\\
END.
\end{proof}


\textbf{Inductive step}: if \textbf{(*)} is true for $j-1$ we can proof that it is true for $J$.\\
Program instruction: $M[2^{j}K] = M[2^{j}K] + M[2^{j}K - 2^{j-1}]$\\
$M[2K \cdot 2^{j-1}] = M[2K \cdot 2^{j-1}] + \dots + M[2^{j-1}(2k-1)+1]$\\
$M[2^{j-1}(2k-1)] \overset{(*)}{=} M[2^{j-1}(2k-1)] + \dots + M[2^{j-1}(2K-2)+1]$\\
$M[2^{j}K] \overset{(*) is true}{=} M[2^{j}K] + \dots + M[2^{j-1}(2k-1)+1]$\\

\textbf{Evaluation of the summation algorithm}

$P(n) = n/2$\\
$T(n, n/2) = 4 \log{n}$ micro-instruction: LD, LD, ADD, ST\\
And if $n$ is not a power of 2?\\
We stretch the input to the nearest power of 2.\\
So we have:\\ 
$P(n) = 2n/2 = 2$\\
$T(n, n) = 4 \log{2n} \leq 5 \log{n}$ \\

\textbf{Efficiency}
$E(n,n) = \frac{n-1}{n \cdot 5 \log{n}} \sim \frac{n}{5 \log{n}} \rightarrow 0 (slowly)$\\
Since the processor are a bit wasted we apply Wyllie: $P(n) = o(n)$ to have $E \rightarrow K \neq 0$\\

\textbf{Apply Wyllie:}\\
$n$ number,\\
$p$ processors,\\
$\Delta = n/p$ to sum.\\

1 parallel step:\\
$M[K \Delta] = M[K \Delta] + \dots + M[(K-1) \Delta+1] \ \ 1 \leq K \leq P$\\
Sunsequent parallel steps:\\
$M[\Delta], M[2 \Delta] + \dots + M[P \Delta]$\\
That store in $M[P \Delta] = M[n] = \sum_i{M[i]}$\\
Correctness: OK\\
Evaluation:\\
$p(n) = P$\\
$T(n, p) = T(1 step) + T(subsequent \ steps) = n/p + 5 \log{p}$\\
$E(n,p) = \frac{n-1}{P(n/p + 5 \log{p}} = \frac{n-1}{n + 5p \log{P}} \rightarrow K \neq 0$\\
$E(n,p) \simeq n/2n \rightarrow 1/2$\\
We would like to have $p \log{P} \leadsto P = \frac{n}{\log{5}}$\\
In fact: $\frac{n}{5 \log{n}} \left(\log{\frac{n}{(\log{n})5}}\right)$\\
$ = \frac{n}{5 \log{n}}\left(\log{n} - \log{5} - \log{\log{n}}\right)$\\
$ = n/5\left(1 - \frac{\log{5}}{\log{n}} - \frac{\log{\log{n}}}{\log{n}}\right) \sim n/5$\\

\textbf{Recap:}\\
$p(n) = \frac{n}{5 \log{n}}$\\
$T(n, p(n)) = 5\log{n} + 5\log{n} - \dots \leq 10\log{n}$\\
Wyllie: [Use a sublinear number of processors and keep a logarithmic time]

\textbf{Can we do better for sum algorithm?}\\
Lower bound\\
\begin{itemize}
 \item for sum algorithm we can see a parallel algorithm using a tree:\\
 leaves: numbers to sum\\
 leyers: parallel step\\
 layer with the most nodes: number of processors\\
 height of the tree: time\\
\end{itemize}

\subsubsection{Sum as a schema for other problems}

\textbf{OP iterata} where op is associative.\\
\textbf{input:} $M[1], \dots, M[n]$\\
\textbf{output:} $OP_i \ M[1] \rightarrow M[n]$\\
For example: $OP = +, *, \bigwedge, \bigvee, \oplus, min, max, \cdot, \dots$\\
We have a efficiency solution in parallel:\\
$P = O(\frac{n}{\log{n}}$ = $T = O(\log{n})$ and with modern PRAM models we can have a constant time.\\

\subsection{$\wedge-iterate$}

Problem $\wedge-iterate$: $M[n] = \bigwedge_{i}{M[i]}$ 

\begin{algorithm}[H]
 \SetAlgoLined
 %\KwIn{}
 %\KwResult{}
 \For{$1 \leq K \leq n$ /*parallelize*/} {
  \If{$M[K] = 0$} {   
   $M[n] = 0$ 
  }
 }
 %\Return{}
 \caption{$\wedge-iterate$}
\end{algorithm}

Note:\\
$CW$ with common politic\\
$p(n) = n$\\
$T(n, n) = 3$\\
$E(n, n) = \frac{n-1}{n \cdot 3} \rightarrow 1/3$\\

Sum as a subproblem of other problems:
\begin{itemize}
 \item Inner product of vectors
 \item Product matrix-vector
 \item Product matrix-matrix
 \item Power of matrix
\end{itemize}



\subsection{Inner product of vectors}

Sequential time = $2n - 1$\\

\textbf{Input:} $x,y \in \mathbb{N}^n$\\
\textbf{Output:} $\langle x,y \rangle = \sum_{i=1}^{n}{x_i \cdot y_i}$\\

\textbf{I phase:}\\
I step $\Delta$ sequence products for each processor and sequential sum.\\
\textbf{II phase:}\\
Sum of $P$ number in parallel.\\

\textbf{For sum:}\\
$p = C_1 \frac{n}{\log{n}} \Rightarrow  t_{II} = C_2 \log{n}$ \textbf{II pahse} \\
$p \simeq \frac{n}{\log{n}} \Rightarrow \Delta = n/p = \log{n} \Rightarrow t_I = C_3 \log{n}$ \textbf{I pahse}

\textbf{Remark}\\
$\langle x,y \rangle$ cost: $p \simeq \frac{n}{\log{n}} \Rightarrow t = t_I + t_{II} \simeq \log{n}$\\
$E \sim \frac{2n-1}{\frac{n}{\log{n}} \cdot \log{n}} \rightarrow C \neq 0 (constant)$

\subsection{Product matrix-vector}

Sequential time = $n(2n-1) = 2n^2 - n$

\textbf{Input:} $A \in \mathbb{N}^{n \cdot n}, x \in \mathbb{N}^{n}$\\
\textbf{Output:} $A \cdot x$\\
\textbf{Idea:} using the module $\langle \dots,\dots \rangle$ in parallel $n$ times.\\
\textbf{Alert:} vector $x$ is accessed simultaneously by $\langle \dots,\dots \rangle$ modules $\Rightarrow$ CREW\\

\textbf{Performance:}\\
$p \simeq n \cdot \frac{n}{\log{n}} \Rightarrow T(n, p(n)) \simeq \log{n}$\\
$E(n,p(n)) \simeq  \frac{n^2}{\frac{n^2}{\log{n}} \cdot \log{n}} \rightarrow C \neq 0 (constant)$

\subsection{Product matrix-matrix}

Sequential time = $n^{2.80}$ (Strassen)

\textbf{Input:} $A,B \in \mathbb{N}^{n \cdot n}$\\
\textbf{Output:} $A \cdot B$\\
\textbf{Idea:} using $n^2$ module $\langle \dots,\dots \rangle$ in parallel\\
\textbf{Alert:} each row of $a$ and column of $B$ is accessed simultaneously $\Rightarrow$ CREW\\

\textbf{Performance:}\\
$p \simeq n^2 \cdot \frac{n}{\log{n}} \Rightarrow T(n, p(n)) \simeq \log{n}$\\
$E(n,p(n)) \simeq  \frac{n^{2.80}}{\frac{n^3}{\cancel{\log{n}}} \cdot \cancel{\log{n}}} = \frac{n^{2.80}}{n^3} \rightarrow 0 (slowly)$

\subsection{Power of matrix}

Sequential time = $n^{2.80} \cdot \log{n}$

(*) note that is a iterative product of the same matrix.

\textbf{Input:} $A \in \mathbb{N}^{n \cdot n}$\\
\textbf{Output:} $A^n, n = 2^k$\\
\textbf{(*) sequencially}
\begin{lstlisting}[]
for i=1 to logn do
 A = A * A
\end{lstlisting}
\textbf{Parallel}\\
$\log{n}$ times the product $A \cdot A$. It is like sequencial where replace $\cdot$ between $a$ to sequentialize to parallel $\Rightarrow$ CREW\\

\textbf{Performance:}\\
$p(n) = \frac{n^3}{\log{n}} \Rightarrow T(n, p(n)) = \log{n} \cdot \log{n} = \log^2{n}$\\
$E(n,p(n)) \simeq  \frac{n^{2.80} \cdot \log{n}}{\frac{n^3}{\log{n}} \cdot \log^2{n}} = \frac{n^{2.80}}{n^3} \rightarrow 0 (slowly)$

\subsection{Prefix sum}

Sequential time = $n - 1$

\textbf{Input:} $M[1], M[2], \dots, M[n]$\\
\textbf{Output:} $\sum_{i_1}^{k} M[i] \rightarrow M[k]$ where $1 \leq k \leq n$\\
We assume $n$ power of 2\\
\textbf{Sequencially}
\begin{lstlisting}[]
for k=2 to n do
 M[k] = M[k] + M[k-1]
\end{lstlisting}

\textbf{Parallel solution}\\
Solve with \textbf{parallel sum} whole possibile prefix.\\\\
\textbf{Problems:}
\begin{itemize}
 \item It is not EREW but I can reduce the problem
 \item I have a CREW algorithm on PRAM with:\\
 $p(n) \leq (n-1) \cdot \frac{n}{\log{n}} \simeq \frac{n^2}{\log{n}}$\\
 $= \sum_{i=2}^{n}\frac{i}{\log{i}} \geq \frac{1}{\log{n}} \sum_{i=2}^{n} i \simeq \frac{n^2}{\log{n}}$\\
 $T(n, p(n)) \simeq \log{n}$ and $E \simeq \frac{n-1}{\frac{n^2}{\log{n}} \cdot \log{n}} \rightarrow 0 (inefficient)$
\end{itemize}

\subsubsection{KOGGE-STONE - pointer doubling (1973)}
\textbf{Idea:} we establish a bond between numbers, each processor deals with a binding and do sum in this way...\\
position : $\dots i \dots \ j \dots$\\
numbers: $\dots m \dots k \dots$\\
$i$ do $m+k$ and put the result in $j$\\
\begin{remark}
 no element has a successor
\end{remark}

\textbf{Some questions:}
\begin{itemize}
 \item How many numbers without succesors are generated at j-th step?\\
 $2^j$
 \item How many steps does the algorithm take?\\
 $\log{n} : 2^j = n \Rightarrow \log{n}$
 \item How many processor are active at j-th step?\\
 $1 \leq k \leq n - 2^{j-1}$
 \item Let $S[k]$ the position $M[k]$ successors, how can I initialize $S$?\\
 $S[k] = k + 1$ and $S[n] = 0$
 \item Given $p_k$ what istruction on M should it execute?\\
 $M[k] + M[S[k]] \rightarrow M[S[k]]$\\
 how should it update $S$?\\
 $S[k] = (S[k]==0 ? 0 : S[S[k]]$ 
\end{itemize} 

\textbf{Parallel algorithm ($M$ and $S$ already given)}\\
\begin{algorithm}[H]
 \SetAlgoLined
 \KwIn{$M[1], M[2], \dots, M[n]$}
 \KwResult{$\sum_{i_1}^{k} M[i] \rightarrow M[k]$ where $1 \leq k \leq n$}
 \For{$j=1 \ to \ \log{n}$} {
  \For{$1 \leq k \leq n-2^j$ /*parallelize*/} {   
   $M[S[k]] = M[k] + M[S[k]]$ \\
   $S[k] = (S[k]==0 ? 0 : S[S[k]])$ 
  }
 }
 %\Return{}
 \caption{Prefix sum - Kogge Stone}
\end{algorithm}

\textbf{Correctness:}
\begin{itemize}
 \item is EREW PRAM\\
 $p_k$ works on $M[k]$ and $M[S[k]]$:\\
 if $i \neq j \Rightarrow S[i] \neq \S[j]$ (they have successors except the case $S[i] = \S[j] = 0$)
 \item 
 \begin{proof}
 $M[k] = \sum_{i=i}^{k}M[i]$ where $1 \leq k \leq n$\\
 Work on j-th property.\\
  \begin{equation}
   M[t] = 
    \begin{cases}
      M[t] + \dots + M[1] where t \leq 2^j\\
      M[t] + \dots + M[t - 2^k + 1] where t \geq 2^j
    \end{cases}\,.
  \end{equation}
 
 \end{proof} 
\end{itemize}


%\input{chapters/1-apx_classi_greedy.tex}
%\input{chapters/2-apx_pricing.tex}
%\input{chapters/3-probability.tex}
%\input{chapters/4-teo.tex}
%\input{chapters/5-str.tex}


\section{Test Graph}

\begin{tikzpicture}[node distance={15mm}, thick, main/.style = {draw, circle}] 
\node[main] (1) {$x_1$}; 

\node[main] (2) [above right of=1] {$x_2$};
\node[main] (3) [below right of=1] {$x_3$}; 
\node[main] (4) [above right of=3] {$x_4$};
\node[main] (5) [above right of=4] {$x_5$}; 
\node[main] (6) [below right of=4] {$x_6$};

\draw (2) -- (4);
\draw[->] (1) -- (2);
\draw[->] (1) -- (3);

\draw (1) to (5);
\draw (1) to [out=135,in=90,looseness=1.5] (5);

\draw (1) to [out=180,in=270,looseness=5] (1);

\draw[->] (6) -- node[midway, above right, sloped, pos=1] {+1} (4);

\end{tikzpicture} 

\begin{tikzpicture}[node distance={15mm}, thick, main/.style = {draw, circle}] 
\node[main] (1) {$x_1$}; 
\node[main] (2) [above right of=1] {$x_2$}; 
\node[main] (3) [below right of=1] {$x_3$}; 
\node[main] (4) [above right of=3] {$x_4$}; 
\node[main] (5) [above right of=4] {$x_5$}; 
\node[main] (6) [below right of=4] {$x_6$}; 
\draw[->] (1) -- (2); 
\draw[->] (1) -- (3); 
\draw (1) to [out=135,in=90,looseness=1.5] (5); 
\draw (1) to [out=180,in=270,looseness=5] (1); 
\draw (2) -- (4); 
\draw (3) -- (4); 
\draw (5) -- (4); 
\draw[->] (5) to [out=315, in=315, looseness=2.5] (3); 
\draw[->] (6) -- node[midway, above right, sloped, pos=1] {+1} (4); 
\end{tikzpicture} 

\end{document}