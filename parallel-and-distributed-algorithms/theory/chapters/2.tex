\section{Time}

In both cases, parallel algorithm and distributed algotithm the crucial resource is time.\\

Formal definition:\\

$T(x) = $ number of elementary operations on $x$ ($x$ is the instance)\\
$t(n) = max\{T(x) \ | \ x \in \sum_{}^{n} \} \ | \  n \ is \ the \ length \ of \ the \ input$

\begin{remark}
Often we will not be interested in a precise evaulation of $T(n)$, but in its growth rate. We will use $O \ \Omega \ \Theta \ $
\end{remark}

\begin{definition}
Be $f,g : \mathbb{N} \rightarrow \mathbb{N}$ two natural function. We can say that:\\
- $f(n) = O(g(n))$ iff $\exists$ a constant $c > 0$ and $n_0 \in \mathbb{N} \ | \ f(n) \leq c*g(n) \ \forall \ n \geq n_0$\\
- $f(n) = \Omega(g(n))$ iff $\exists$ a constant $c > 0$ and $n_0 \in \mathbb{N} \ | \ f(n) \geq c*g(n) \ \forall \ n \geq n_0$\\
- $f(n) = \Theta(g(n))$ iff $\exists$ two constant $c1, c2 > 0$ and $n_0 \in \mathbb{N} \ | \ c1*g(n) \leq \ f(n) \ \leq c2*g(n) \ \forall \ n \geq n_0$\end{definition}

\begin{enumerate}
 \item $t(n)$ is evaluated in a particular model of computation
 \item the cost criterion must be chosen: uniform of logarithmic
\end{enumerate}

We count the primitives that our model of computation gives us.

\textbf{Palindrome}

\begin{algorithm}[H]
 \SetAlgoLined
 \KwIn{$x \in \{0,1\}^*$}
 \KwResult{$x$ is palindrome}
 \For{$i=0,j=|x|; \ i<j \ \& \ x_i=x_j; \ i++, j--$}{}
 \Return{$i \geq j$}
 \caption{Palindrome}
\end{algorithm}

This program:

\begin{itemize}
 \item on RAM has $t(n) == O(n)$
 \item on DTM (deterministic Turing machine) has $t(n) == O(n^2)$
\end{itemize}

\textbf{Factorial}

\begin{algorithm}[H]
 \SetAlgoLined
 \KwIn{$x \in \{0,1\}^*$}
 \KwResult{$n!$}
 $k \gets 1$\\
 \For{$i=1; \ i \leq x; \ k=k*i, i++$}{}
 \Return{$k$}
 \caption{Factorial}
\end{algorithm}

This program:\\

\begin{itemize}
 \item using multiplication as an elementary operation of RAM\\
$t(\lg{n}) = n$
 \item on DTM we should write the result in binary is $t(\lg{n}) \geq \lg{n^n} \ = \ n \lg{n}$
\end{itemize}

\textbf{Uniform cost criterion}\\
Elementary operations needed one unit time\\

\textbf{Logarithmic cost criterion}\\
Each elementary operations has a cost that depends on the number of bits of the operands

\textbf{Complexity theory}

$t(n)$ is:\\
\begin{itemize}
 \item logarithmic when is $O(\lg{n})$
 \item polylogarithmic when is $O(\lg^k{n})$
 \item linear when is $O(n)$
 \item polynomial when is $O(n^k) \ \forall k>0$ 
 \item exponential when is NOT $O(n^k) \ \forall k>0$ 
\end{itemize}

\textbf{Efficiency concept}

\begin{definition}
A problem is efficienty solved in time if and only if is solved in a DTM in polynomial time.
\end{definition}

$P$ = decision problem class solved in polynomial time\\
$FP$ = general problem class solved in polynomial time\\
$NP$ = decision problem class solved in polynomial time on nDTM (non-deterministic Turing machine)

\begin{remark}
 the asymptotic evaluation of times can hide constants that make the difference in practice\\
  i.e., sometimes we prefer quick-sort ($O(n^2)$) over merge-sort ($O(n\lg{n})$)\\
 the degree of the polynomial must be low\\
  i.e., $n^{1000}$ on small instance is worse than exponential algorithms\\
\end{remark}

\section{Parallel Algorithms}

\textbf{Problems}:

\begin{enumerate}
 \item \textbf{synthesis}:\\
  How to build algorithms? We can draw inspiration from sequential algoithms, but we need to have innovative ideas.
 \item \textbf{evaluation}:\\
 How to evaluate performance?\\
  - $time$\\
  - $hardware$ = \# processor\\
 What is efficient?\\
  - $E$ = efficiency\\
   It is usefull to have a ratio between $time$ and $hardware$
 \item \textbf{universality}: \\
  Am I able to define a class of problems for efficient parallel algorithm?\\
  Sequential = $FP$\\
  Parallel = $NC$\\
  $NC \rightarrow$ problem solved by faster parallel algorithm with:\\
   - polylogarithmic time\\
   - polynomial hardware
  \begin{proof}
  $NC \subseteq FP$\\
  To obtain a sequential algorithm starting from a parallel one, it is enough to simulate "with a processor" the work of the processors involved in sequence.
  \end{proof}
\end{enumerate}
