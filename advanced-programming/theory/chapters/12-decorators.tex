 \section{Decorators}

How to Silently Extend Classes (Part 2)

\subsection{Class Extensions through Decorators}

\subsubsection{What’s a Decorator?}

Decoration is a way to specify management code for functions and classes.

- Decorators themselves take the form of callable objects (e.g., functions) that process other callable objects.

Python decorators come in two related flavors:

Function decorators do name rebinding at function definition time, providing a layer of logic that can manage functions and methods, or later calls to them.

Class decorators do name rebinding at class definition time, providing a layer of logic that can manage classes, or the instances created by calling them later

In short, decorators provide a way to automatically run code at the end of function and class definition statements.


\subsubsection{Function Decorators}

\begin{lstlisting}[language=Python]
def decorator(F): # on @ decoration
	def wrapper(*args): # on wrapped function call
		# Use F and args and then call F(*args)
		print("I’m executing the call {0}{1} ...". \
			format(F.__name__, args))
		return F(*args)
	return wrapper
	
@decorator
def f(x,y):
	print("*** f({0}, {1})".format(x,y))

f(42, 7)
\end{lstlisting}

\begin{lstlisting}[language=Python]
[23:30]cazzola@hymir:~/esercizi-pa>python3 fdecs.py
I’m executing the call f(42, 7) ...
*** f(42, 7)
\end{lstlisting}

\begin{lstlisting}[language=Python]
class wrapper:
	def __init__(self, func): # On @ decoration
		self.func = func
	def __call__(self, *args): # On wrapped calls
		# Use func and args and then call func(*args)
		print("I’m executing the call {0}{1} ...". \
			format(self.func.__name__, args))
	return self.func(*args)
	
@wrapper
def f2(x,y,z):
	print("*** f2({0}, {1}, {2})".format(x,y,z))
	
f2("abc",7, ’ß’)
\end{lstlisting}

\begin{lstlisting}[language=Python]
[23:31]cazzola@hymir:~/esercizi-pa>python3 fdecs.py
I’m executing the call f2(’abc’, 7, ’ß’) ...
*** f2(abc, 7, ß)
\end{lstlisting}

\textbf{@decorator f(5,7) ≡ decorator(f)(5,7)}

Note that,
- methods cannot be decorated by function decorators since the self would be associated to the decorator.

\subsubsection{Class Decorators}

\begin{lstlisting}[language=Python]
def decorator(cls): # On @ decoration
	class wrapper:
		def __init__(self, *args): # On instance creation
			print("I’m creating {0}{1} ...".format(cls.__name__, args))
			self.wrapped = cls(*args)
		def __getattr__(self, name): # On attribute fetch
			print("I’m fetching {0}.{1} ...".format(self.wrapped, name))
			return getattr(self.wrapped, name)
		def __setattr__(self, attribute, value): # On attribute set
			print("I’m setting {0} to {1} ...".format(attribute, value))
			if attribute == ’wrapped’: # Allow my attrs
				self.__dict__[attribute] = value # Avoid looping
			else:
				setattr(self.wrapped, attribute, value)
	return wrapper
	
@decorator
class C: # C = decorator(C)
	def __init__(self, x, y): self.attr = ’spam’
	def f(self, a, b): print("*** f({0}, {1})".format(a,b))
\end{lstlisting}

\begin{lstlisting}[language=Python]
[0:06]cazzola@hymir:~/esercizi-pa/decorators>python3
>>> from cdecorators import *
>>> x = C(6, 7)
I’m creating C(6, 7) ...
I’m setting wrapped to <cdecorators.C object at 0xb79eb26c> ...
>>> print(x.attr)
I’m fetching <cdecorators.C object at 0xb79eb26c>.attr ...
spam
>>> x.f(x.attr, 7)
I’m fetching <cdecorators.C object at 0xb79eb26c>.f ...
I’m fetching <cdecorators.C object at 0xb79eb26c>.attr ...
*** f(spam, 7)
\end{lstlisting}

\subsubsection{Function Decorators at Work: Timing}

\begin{lstlisting}[language=Python]
import time
class timer:
	def __init__(self, func):
		self.func = func
		self.alltime = 0
	def __call__(self, *args, **kargs):
		start = time.clock()
		result = self.func(*args, **kargs)
		elapsed = time.clock() - start
		self.alltime += elapsed
		print(’{0}: {1:.5f}, {2:.5f}’.\
			format(self.func.__name__, elapsed, self.alltime))
		return result

@timer
def listcomp(N):
	return [x * 2 for x in range(N)]
@timer
def mapcall(N):
	return list(map((lambda x: x * 2), range(N)))
	
if __name__ == "__main__":
	result = listcomp(5)
	listcomp(50000)
	listcomp(500000)
	listcomp(1000000)
	print(result)
	print(’allTime = {0}’.format(listcomp.alltime))
	print(’’)
	result = mapcall(5)
	mapcall(50000)
	mapcall(500000)
	mapcall(1000000)
	print(result)
	print(’allTime = {0}’.format(mapcall.alltime))
	print(’map/comp = {0}’.format(\
	round(mapcall.alltime / listcomp.alltime, 3)))
\end{lstlisting}

\begin{lstlisting}[language=Python]
[21:06]cazzola@hymir:~/esercizi-pa>python3 timing.py
listcomp: 0.00000, 0.00000
listcomp: 0.03000, 0.03000
listcomp: 0.41000, 0.44000
listcomp: 0.85000, 1.29000
[0, 2, 4, 6, 8]
allTime = 1.29
mapcall: 0.00000, 0.00000
mapcall: 0.07000, 0.07000
mapcall: 0.71000, 0.78000
mapcall: 1.41000, 2.19000
[0, 2, 4, 6, 8]
allTime = 2.19
map/comp = 1.698
\end{lstlisting}

\subsubsection{Class Decorators at Work: Tracer}

\begin{lstlisting}[language=Python]
def Tracer(aClass): # On @ decorator
	class Wrapper:
		def __init__(self, *args, **kargs): # On instance creation
			self.fetches = 0
			self.wrapped = aClass(*args, **kargs) # Use enclosing scope name
		def __getattr__(self, attrname):
			print(’Trace: ’ + attrname) # Catches all but own attrs
			self.fetches += 1
			return getattr(self.wrapped, attrname) # Delegate to wrapped obj
		return Wrapper

@Tracer
class Person: # Person = Tracer(Person)
	def __init__(self, name, hours, rate): # Wrapper remembers Person
		self.name = name
		self.hours = hours
		self.rate = rate
	def pay(self): # Accesses outside class traced
		return self.hours * self.rate # In-method accesses not traced
\end{lstlisting}

\begin{lstlisting}[language=Python]
[12:59]cazzola@hymir:~/esercizi-pa/decorators>python3
>>> from tracer import *
>>> bob = Person(’Bob’, 40, 50)
>>> print(bob.name) # bob is a Wrapper to a Person
Trace: name
Bob
>>> print(bob.pay())
Trace: pay
2000
>>> sue = Person(’Sue’, rate=100, hours=60)
>>> print(sue.name) # sue is a different Wrapper
Trace: name
Sue
>>> print(sue.pay())
Trace: pay
6000
>>> print(bob.name) # bob has a different state
Trace: name
Bob
>>> print(bob.pay())
Trace: pay
2000
>>> print([bob.fetches, sue.fetches])
[4, 2]
\end{lstlisting}


\subsubsection{Class Decorators at Work: Singleton}

\begin{lstlisting}[language=Python]
class singleton:
	def __init__(self, aClass):
		self.aClass = aClass
		self.instance = None
	def __call__(self, *args):
		if self.instance == None:
			self.instance = self.aClass(*args) # One instance per class
		return self.instance

@singleton # Person = singleton(Person)
class Person: # Rebinds Person to onCall
	def __init__(self, name, hours, rate): # onCall remembers Person
		self.name = name
		self.hours = hours
		self.rate = rate
	def pay(self):
		return self.hours * self.rate
@singleton # Spam = singleton(Spam)
class Spam: # Rebinds Spam to onCall
	def __init__(self, val): # onCall remembers Spam
		self.attr = val
\end{lstlisting}

\begin{lstlisting}[language=Python]
[21:29]cazzola@hymir:~/esercizi-pa/decorators>python3
>>> from singleton import *
>>> bob = Person(’Bob’, 40, 10)
>>> print(bob.name, bob.pay())
Bob 400
>>> sue = Person(’Sue’, 50, 20)
>>> print(sue.name, sue.pay())
Bob 400
>>> X = Spam(42)
>>> Y = Spam(99)
>>> print(X.attr, Y.attr)
42 42
\end{lstlisting}

\subsubsection{Class Decorators at Work: Privateness}

\begin{lstlisting}[language=Python]
traceMe = False
def trace(*args):
	if traceMe: print(’[’ + ’ ’.join(map(str, args)) + ’]’)
	
def Private(*privates): # privates in enclosing scope
	def onDecorator(aClass): # aClass in enclosing scope
		class onInstance: # wrapped in instance attribute
			def __init__(self, *args, **kargs):
				self.wrapped = aClass(*args, **kargs)
			def __getattr__(self, attr): # My attrs don’t call getattr
				trace(’get:’, attr) # Others assumed in wrapped
				if attr in privates:
					raise TypeError(’private attribute fetch: ’ + attr)
				else:
					return getattr(self.wrapped, attr)
			def __setattr__(self, attr, value): # Outside accesses
				trace(’set:’, attr, value) # Others run normally
				if attr == ’wrapped’: # Allow my attrs
					self.__dict__[attr] = value # Avoid looping
				elif attr in privates:
					raise TypeError(’private attribute change: ’ + attr)
				else:
					setattr(self.wrapped, attr, value) # Wrapped obj attrs
			return onInstance # Or use __dict__
		return onDecorator
\end{lstlisting}

\begin{lstlisting}[language=Python]
[22:05]cazzola@hymir:~/esercizi-pa/decorators>python3
>>> from private import *
>>> traceMe = True
>>> @Private(’data’, ’size’)
... class Doubler:
... def __init__(self, label, start):
... self.label = label # Accesses inside the subject class
... self.data = start # Not intercepted: run normally
... def size(self):
... return len(self.data) # Methods run with no checking
... def double(self): # Because privacy not inherited
... for i in range(self.size()):
... self.data[i] = self.data[i] * 2
... def display(self):
... print(’{0} => {1}’.format(self.label, self.data))
>>> X = Doubler(’X’, [1, 2, 3])
>>> print(X.label) # Accesses outside subject class
X
>>> X.display(); X.double(); X.display() # Intercepted: validated, delegated
X => [1, 2, 3]
X => [2, 4, 6]
>>> print(X.size()) # prints "TypeError: private attribute fetch: size"
Traceback (most recent call last):
File "<stdin>", line 1, in <module>
File "private.py", line 19, in __getattr__
raise TypeError(’private attribute fetch: ’ + attr)
TypeError: private attribute fetch: size
>>> X.data = [1, 1, 1]
Traceback (most recent call last):
File "<stdin>", line 1, in <module>
File "private.py", line 27, in __setattr__
raise TypeError(’private attribute change: ’ + attr)
TypeError: private attribute change: data
\end{lstlisting}