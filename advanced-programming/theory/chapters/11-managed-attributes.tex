\section{Managed Attributes}

How to silently extend classes

\subsection{Class Extensions through Managed Attributes}

\subsubsection{Case Study: Account}

Let us consider the classic implementation for the account class

\begin{lstlisting}[language=Python]
class account:
	def __init__(self, initial_amount):
		self.amount = initial_amount
	def balance(self):
		return self.amount
	def withdraw(self, amount):
		self.amount -= amount
	def deposit(self, amount):
		self.amount += amount
if __name__ == "__main__":
	a = account(1000)
	print("The current balance is {0}".format(a.balance()))
	a.withdraw(100)
	a.deposit(750)
	print("The current balance is {0}".format(a.balance()))
	a.withdraw(3000)
	print("The current balance is {0}".format(a.balance()))
\end{lstlisting}

\begin{lstlisting}[language=Python]
[23:15]cazzola@hymir:~/esercizi-pa/managed>python3 account.py
The current balance is 1000
The current balance is 1650
The current balance is -1350
\end{lstlisting}

What’s about adding a functionality w/o polluting its code?
- key concept: separation of concerns

\subsubsection{Inserting Code to Run on Attribute Access}
A clean approach is to (automatically) execute extra code when an attribute is accessed.

Python provides 3 Approaches:
- properties
- descriptor protocol (deja vu)
- operator overloading

\subsubsection{Properties: To Avoid Red Balances}

\begin{lstlisting}[language=Python]
import account

class safe_account(account.account):
	def __init__(self, initial_amount):
		self._amount = initial_amount
	def save_get(self):
		return self._amount
	def save_set(self, amount):
		assert amount > 0, ’Not admitted operation: the final balance ({0}) MUST be positive’.format(amount)
		self._amount=amount
		amount = property(save_get, save_set, None, "Managed balance against excessive withdrawals")

if __name__ == "__main__":
	a = safe_account(1000)
	print("The current balance is {0}".format(a.balance()))
	a.withdraw(100)
	a.deposit(750)
	print("The current balance is {0}".format(a.balance()))
	a.withdraw(3000)
	print("The current balance is 0".format(a.balance()))
\end{lstlisting}

\begin{lstlisting}[language=Python]
[23:31]cazzola@hymir:~/esercizi-pa/managed>python3 account+property.py
The current balance is 1000
The current balance is 1650
Traceback (most recent call last):
File "account+property.py", line 19, in <module>
a.withdraw(3000)
File "/home/cazzola/esercizi-pa/managed/account.py", line 7, in withdraw
self.amount -= amount
File "account+property.py", line 9, in save_set
assert amount > 0, ’Not admitted operation: the final balance ({0}) MUST be positive’.format(amount)
AssertionError: Not admitted operation: the final balance (-1350) MUST be positive
\end{lstlisting}

\subsubsection{Properties: To Dynamically Calculate the Balance}

\begin{lstlisting}[language=Python]
class account_with_calculated_balance:
	def __init__(self, initial_amount):
		self._deposits = initial_amount
		self._withdrawals = 0
	def deposit(self, amount):
		self._deposits += amount
	def withdraw(self, amount):
		self._withdrawals += amount
	def calculated_balance(self):
		return self._deposits-self._withdrawals
	def zeroing_balance(self):
		self._deposits = 0
		self._withdrawals = 0
	balance = property(calculated_balance, None, zeroing_balance, "Calculate Balance")

if __name__ == "__main__":
	a = account_with_calculated_balance(1000)
	print("The current balance is {0}".format(a.balance))
	a.withdraw(100)
	a.deposit(750)
	print("The current balance is {0}".format(a.balance))
	a.withdraw(3000)
	print("The current balance is {0}".format(a.balance))
	del a.balance
	print("The current balance is {0}".format(a.balance))
\end{lstlisting}

\begin{lstlisting}[language=Python]
[23:57]cazzola@hymir:~/esercizi-pa/managed>python3 account+property2.py
The current balance is 1000
The current balance is 1650
The current balance is -1350
The current balance is 0
\end{lstlisting}

\subsubsection{Descriptor Protocol: To Avoid Red Balances}

\begin{lstlisting}[language=Python]
import account

class safe_descriptor:
	"""Managed balance against excessive withdrawals"""
	def __get__(self, instance, owner):
		return instance._amount
	def __set__(self, instance, amount):
		assert amount > 0, ’Not admitted operation: the final balance ({0}) MUST be positive’.format(amount)
		instance._amount=amount
class safe_account(account.account):
	def __init__(self, initial_amount):
		self._amount = initial_amount
		amount = safe_descriptor()

if __name__ == "__main__":
	a = safe_account(1000)
	print("The current balance is {0}".format(a.balance()))
	a.withdraw(100)
	a.deposit(750)
	print("The current balance is {0}".format(a.balance()))
	a.withdraw(3000)
	print("The current balance is 0".format(a.balance()))
\end{lstlisting}

\begin{lstlisting}[language=Python]
[23:59]cazzola@hymir:~/esercizi-pa/managed>python3 account+descriptors.py
The current balance is 1000
The current balance is 1650
Traceback (most recent call last):
File "account+descriptors.py", line 22, in <module>
a.withdraw(3000)
File "/home/cazzola/esercizi-pa/managed/account.py", line 7, in withdraw
self.amount -= amount
File "account+descriptors.py", line 8, in __set__
assert amount > 0, ’Not admitted operation: the final balance ({0}) MUST be positive’.format(amount)
AssertionError: Not admitted operation: the final balance (-1350) MUST be positive
\end{lstlisting}

\subsubsection{Descriptor Protocol: To Dynamically Calculate the Balance}

\begin{lstlisting}[language=Python]
class balance_descriptor:
	"""Calculate Balance"""
	def __get__(self, instance, owner):
		return instance._deposits-instance._withdrawals
	def __delete__(self, instance):
		instance._deposits = 0
		instance._withdrawals = 0
class account_with_calculated_balance:
	def __init__(self, initial_amount):
		self._deposits = initial_amount
		self._withdrawals = 0
	def deposit(self, amount):
		self._deposits += amount
	def withdraw(self, amount):
		self._withdrawals += amount
		balance = balance_descriptor()
if __name__ == "__main__":
	a = account_with_calculated_balance(1000)
	print("The current balance is {0}".format(a.balance))
	a.withdraw(100)
	a.deposit(750)
	print("The current balance is {0}".format(a.balance))
	a.withdraw(3000)
	print("The current balance is {0}".format(a.balance))
	del a.balance
	print("The current balance is {0}".format(a.balance))
\end{lstlisting}

\begin{lstlisting}[language=Python]
[0:05]cazzola@hymir:~/esercizi-pa/managed>python3 account+descriptors2.py
The current balance is 1000
The current balance is 1650
The current balance is -1350
The current balance is 0
\end{lstlisting}

\subsubsection{Operator Overloading Protocol}

\begin{itemize}
	\item \_\_getattr\_\_ is run for fetches on undefined attributes
	\item \_\_getattribute\_\_ is run for fetches on every attribute, so when using it you must be cautious to avoid recursive loops by passing attribute accesses to a superclass.
	\item \_\_setattr\_\_ try to guess
	\item \_\_delattr\_\_ is run for deletion on every attribute
\end{itemize}

\subsubsection{Operator Overloading Protocol: To Avoid Red Balances}

\begin{lstlisting}[language=Python]
import account

class safe_account(account.account):
	def __setattr__(self, attr, amount):
		assert amount > 0, ’Not admitted operation: the final balance ({0}) MUST be positive’.format(amount)
		self.__dict__[attr] = amount

if __name__ == "__main__":
	a = safe_account(1000)
	print("The current balance is {0}".format(a.balance()))
	a.withdraw(100)
	a.deposit(750)
	print("The current balance is {0}".format(a.balance()))
	a.withdraw(3000)
	print("The current balance is 0".format(a.balance()))
\end{lstlisting}

\begin{lstlisting}[language=Python]
[0:29]cazzola@hymir:~/esercizi-pa/managed>python3 account+overloading.py
The current balance is 1000
The current balance is 1650
Traceback (most recent call last):
File "account+overloading.py", line 16, in <module>
a.withdraw(3000)
File "/home/cazzola/esercizi-pa/managed/account.py", line 7, in withdraw
self.amount -= amount
File "account+overloading.py", line 7, in __setattr__
assert amount > 0, ’Not admitted operation: the final balance ({0}) MUST be positive’.format(amount)
AssertionError: Not admitted operation: the final balance (-1350) MUST be positive
\end{lstlisting}

\subsubsection{Operator Overloading: To Dynamically Calculate the Balance}

\begin{lstlisting}[language=Python]
class account_with_calculated_balance:
	def __init__(self, initial_amount):
		self._deposits = initial_amount
		self._withdrawals = 0
	def deposit(self, amount):
		self._deposits += amount
	def withdraw(self, amount):
		self._withdrawals += amount
	def __getattr__(self, attr):
		if attr == ’balance’:
			return self._deposits-self._withdrawals
		else: raise AttributeError(attr)
	def __delattr__(self, attr):
		if attr == ’balance’:
			self._deposits = 0
			self._withdrawals = 0
		else: raise AttributeError(attr)

if __name__ == "__main__":
	a = account_with_calculated_balance(1000)
	print("The current balance is {0}".format(a.balance))
	a.withdraw(100)
	a.deposit(750)
	print("The current balance is {0}".format(a.balance))
	a.withdraw(3000)
	print("The current balance is {0}".format(a.balance))
	del a.balance
	print("The current balance is {0}".format(a.balance))
\end{lstlisting}

\begin{lstlisting}[language=Python]
[0:38]cazzola@hymir:~/aux_work/projects/python/esercizi-pa/managed>python3 account+overloading2.py
The current balance is 1000
The current balance is 1650
The current balance is -1350
The current balance is 0
\end{lstlisting}

\subsubsection{\_\_getattr\_\_ vs \_\_getattribute\_\_}

\begin{lstlisting}[language=Python]
class GetAttr:
	attr1 = 1
	def __init__(self):
		self.attr2 = 2
	def __getattr__(self, attr):
		print(’get: ’ + attr)
		return 3

class GetAttribute(object):
	attr1 = 1
	def __init__(self):
		self.attr2 = 2
	def __getattribute__(self, attr):
		print(’get: ’ + attr)
		if attr == ’attr3’:
			return 3
		else:
			return object.__getattribute__(self, attr)
\end{lstlisting}

\begin{lstlisting}[language=Python]
[0:51]cazzola@hymir:~/esercizi-pa/managed>python3
>>> from GetAttr import GetAttr
>>> X=GetAttr()
>>> print(X.attr1)
1
>>> print(X.attr2)
2
>>> print(X.attr3)
get: attr3
3

[0:58]cazzola@hymir:~/esercizi-pa/managed>python3
>>> from GetAttribute import GetAttribute
>>> X = GetAttribute()
>>> print(X.attr1)
get: attr1
1
>>> print(X.attr2)
get: attr2
2
>>> print(X.attr3)
get: attr3
3
\end{lstlisting}
