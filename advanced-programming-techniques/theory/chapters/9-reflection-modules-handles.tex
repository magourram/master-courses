\section{Reflection vs Modules and Handles}

\subsection{Modules and Reflection}

\subsubsection{Private and Reflective Access}

\begin{lstlisting}[language=Java]
package tsp.module.employee;
import tsp.module.reflection.* ;
class Employee implements SmartFieldAccess {
	private String name;
	private String surname;
	public Employee(String n, String s) { this.name = n; this.surname = s; }
	public String toString() {return "Employee: "+this.name+" "+this.surname;}
}
\end{lstlisting}

\begin{lstlisting}[language=Java]
package tsp.module.reflection ;
import java.lang.reflect.*;
public interface SmartFieldAccess {
	default public Object instVarAt(String name) throws Exception {
		Field f = this.getClass().getDeclaredField(name);
		f.setAccessible(true);
		if (!Modifier.isStatic(f.getModifiers())) return f.get(this);
		return null;
	}

	default public void instVarAtPut(String name, Object value) throws Exception {
		Field f = this.getClass().getDeclaredField(name);
		f.setAccessible(true);
		if (!Modifier.isStatic(f.getModifiers())) f.set(this, value);
	}
}
\end{lstlisting}

\subsubsection{Private and Reflective Access (Cont'd).}

\begin{lstlisting}[language=Java]
package tsp.module.employee ;
public class ReflectiveEmployeeMain {
	public static void main(String[] args) throws Exception {
		Employee angela = new Employee("Angela", "Runedottir");
		System.out.println(angela);
		angela.instVarAtPut("surname", "Odindottir");
		System.out.println(angela);
	}
}
\end{lstlisting}

\begin{lstlisting}[language=Java]
[16:38]cazzola@hymir:~/tsp>java tsp.module.employee.ReflectiveEmployeeMain
Employee: Angela Runedottir
Employee: Angela Odindottir
\end{lstlisting}

\subsubsection{Modules: Whats, Whys and Hows}

Java 9 introduced the Java Module Platform.

Its goals are:

\begin{itemize}
	\item Reliable configuration, to replace the brittle, error-prone classpath mechanism in favor of a programming mechanism to explicitly declare dependencies
	\item Strong encapsulation, to allow a component to declare which of its public types are accessible to other components, and which are not.
\end{itemize}

A module:

\begin{itemize}
	\item groups classes, interfaces and packages
	\item a module-info.java file describes the dependencies between modules
\end{itemize}

\begin{lstlisting}[language=Java]
module com.foo.bar {
	requires org.baz.foo ;
	exports com.foo.bar.abc;
}
\end{lstlisting}

- the module-info.class file is shipped inside a jar file.

\subsubsection{Modules a New Obstacle to Reflection}

\begin{lstlisting}[language=Java]
module tsp.module.employee {
	requires tsp.module.reflection ;
	exports tsp.module.employee ;
}
\end{lstlisting}

\begin{lstlisting}[language=Java]
module tsp.module.reflection {
	exports tsp.module.reflection ;
}
\end{lstlisting}

\begin{lstlisting}[language=Java]
> javac tmr/module-info.java tmr/tsp/module/reflection/SmartFieldAccess.java
> javac --module-path=. tme/module-info.java tme/tsp/module/employee/*.java
> java --module-path . -m tme/tme.ReflectiveEmployeeMain
Employee: Angela Runedottir
Exception in thread "main" java.lang.reflect.InaccessibleObjectException: Unable to
make field private java.lang.String tme.Employee.surname accessible:
module tme does not "opens tme" to module tmr
...
at tmr/tmr.SmartFieldAccess.instVarAtPut(SmartFieldAccess.java:15)
at tme/tme.ReflectiveEmployeeMain.main(ReflectiveEmployeeMain.java:12)
\end{lstlisting}

Where:
- tmr stands for tsp.module.reflection, and
- tme stands for tsp.module.employee

\subsubsection{How to Get around Modules' Strong Encapsulation}

Several things can be done:

\begin{itemize}
	\item To pass -add-opens to the JVM to open the module
	- Isn’t always possible to pass a flag to the JVM, e.g., with frameworks0/tools that activate their own JVM.
	\item to open the
	- tme module to everyone,
	- tme package to everyone, or
	- tme package only to tmr
	\item in the corresponding module-info.java file
	- These options aren’t always applicable, e.g., in case of a too complex
project, when the module-info.java file is unavailable, . . .
\end{itemize}

Let’s see this last option

\subsubsection{Reflection Works Only on Open Modules}

\begin{lstlisting}[language=Java]
module tsp.module.employee {
	requires tsp.module.reflection ;
	exports tsp.module.employee ;
}
\end{lstlisting}

\begin{lstlisting}[language=Java]
module tsp.module.reflection {
	exports tsp.module.reflection ;
}
\end{lstlisting}

\begin{lstlisting}[language=Java]
> javac tmr/module-info.java tmr/tsp/module/reflection/SmartFieldAccess.java
> javac --module-path=. tme/module-info.java tme/tsp/module/employee/*.java
> java --module-path . -m tme/tme.ReflectiveEmployeeMain
Employee: Angela Runedottir
Employee: Angela Odindottir
\end{lstlisting}

Where:
- tmr stands for tsp.module.reflection, and
- tme stands for tsp.module.employee

\subsubsection{Module Graph}

The module system resolves the dependencies among modules by

\begin{itemize}
	\item locating observable modules to fulfill those dependencies, and then
	\item resolves the dependencies of these modules, and so forth
	\item until every dependency of every module is fulfilled.
\end{itemize}

This transitive-closure computation originates a module graph where each directed edge represents a fulfilled dependency among modules (the nodes).

\begin{lstlisting}[language=Java]
[23:40]cazzola@hymir:~/tsp>jdeps --module-path=. -m tme -summary -recursive
tsp.module.employee -> java.base
tsp.module.employee -> tsp.module.reflection
tsp.module.reflection -> java.base
\end{lstlisting}

\subsubsection{Reflecting on Modules}

\begin{lstlisting}[language=Java]
[23:52]cazzola@hymir:~/tsp> jshell --module-path=. --add-modules=tsp.module.employee
--add-exports=tsp.module.employee/tsp.module.employee:jshell
jshell> import tsp.module.reflection.* ;
jshell> import tsp.module.employee.* ;
jshell> Module myClassModule = Employee.class.getModule();
myClassModule ==> module tsp.module.employee
jshell> myClassModule.isNamed();
$1 ==> true
jshell> import java.lang.module.*;
jshell> ModuleDescriptor descriptor = myClassModule.getDescriptor();
descriptor ==> module {name:tsp.module.employee, [mandated ... [tsp.module.reflection]]}
jshell> descriptor.toString();
$2 ==> "module { name: tsp.module.employee,
[mandated java.base (@11), tsp.module.reflection],
exports: [tsp.module.employee],
opens: [tsp.module.employee to [tsp.module.reflection]] }"
jshell> ModuleDescriptor d2 = SmartFieldAccess.class.getModule().getDescriptor() ;
d2 ==> module { name: tsp.module.reflection, [mandated j ...
[tsp.module.reflection] }
jshell> d2.toString();
$3 ==> "module { name: tsp.module.reflection,
[mandated java.base (@11)], exports: [tsp.module.reflection] }"
jshell> descriptor.exports();
$4 ==> [tsp.module.employee]
jshell> descriptor.packages();
$5 ==> [tsp.module.employee]
\end{lstlisting}

\subsubsection{Privates Are Not So Private}

\begin{lstlisting}[language=Java]
import java.lang.invoke.MethodHandles.*;
class Employee implements SmartFieldAccess {
	private final Lookup lookup;
	public Lookup getEmployeeLookup() {return this.lookup;}
	private String name; private String surname;
	public Employee(String n, String s){lookup=MethodHandles.lookup(); name=n; 			surname=s;}
}
\end{lstlisting}

\begin{lstlisting}[language=Java]
import java.lang.invoke.*;
import java.lang.invoke.MethodHandles.*;
public interface SmartFieldAccess {
	default public Object instVarAt(Lookup lookup, String name) throws Exception {
		Class<?> clazz = this.getClass();
		Field f = clazz.getDeclaredField(name);
		MethodHandles.Lookup privateLookup = MethodHandles.privateLookupIn(clazz, lookup);
		VarHandle handle = privateLookup.unreflectVarHandle(f);
		if (!Modifier.isStatic(f.getModifiers())) return handle.get(this);
		return null;
	}
	default public void instVarAtPut(Lookup lookup, String n, Object v) throws Exception{
		Class<?> clazz = this.getClass();
		Field f = clazz.getDeclaredField(n);
		MethodHandles.Lookup privateLookup = MethodHandles.privateLookupIn(clazz, lookup);
		VarHandle handle =  privateLookup.unreflectVarHandle(f);
		if (!Modifier.isStatic(f.getModifiers())) handle.set(this, v);
	}
}
\end{lstlisting}


\subsubsection{Privates Are Not So Private (Cont'd)}

\begin{lstlisting}[language=Java]
package tsp.module.employee ;

public class GlasnotEmployeeMain {
	public static void main(String[] args) throws Exception {
		Employee angela = new Employee("Angela", "Runedottir");
		System.out.println(angela);
		angela.instVarAtPut(angela.getEmployeeLookup(), "surname", "Odindottir");
		System.out.println(angela);
	}
}
\end{lstlisting}

\begin{lstlisting}[language=Java]
> javac tmr/module-info.java tmr/tsp/module/reflection/SmartFieldAccess.java
> javac --module-path=. tme/module-info.java tme/tsp/module/employee/*.java
> java --module-path . -m tme/tme.GlasnotEmployeeMain
Employee: Angela Runedottir
Employee: Angela Odindottir
\end{lstlisting}

where:
- tmr stands for tsp.module.reflection, and
- tme stands for tsp.module.employee

\subsubsection{Variable Handles}

A variable handle is a typed reference to a variable:

\begin{itemize}
	\item VarHandle class provides read/write access to variables
	\item VarHandles are immutable and have no visible state
\end{itemize}

Each VarHandle has:

\begin{itemize}
	\item a variable type T that corresponds to the type of the variable
	\item abstracts the safe access to a memory location
	\item a list of coordinate types CT1. CT2, . . . , CTn
\end{itemize}

\begin{lstlisting}[language=Java]
Factory methods to produce/lookup VarHandles:
MethodHandles.lookup().in(Foo.class).findVarHandle(Foo.class, "i", int.class);
\end{lstlisting}

\begin{lstlisting}[language=Java]
MethodHandles.privateLookupIn(Foo.class, MethodHandles.lookup())
.findVarHandle(Foo.class, "i", int.class);
\end{lstlisting}

\subsubsection{Access modes}
\begin{itemize}
	\item read, e.g., get, getVolatile, ...
	\item write, e.g., set, setOpaque, ...
	\item atomic update, compareAndSet, getAndSet, ...
\end{itemize}