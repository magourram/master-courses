\section{Java Annotations}

\subsection{Java Annotations}

\subsubsection{Meta-Data: What, Why and How}

\textbf{Annotations}, also called \textbf{meta-data}, are:

Data about the program execution can be introspected as well:
\begin{itemize}
	\item structured data that describe, explain, locate or make easier to retrieve, use or manage an information resource
\end{itemize}

Annotations are information about information that are interpreted at some point by code or data analysis tools.

Annotations can be used to:
\begin{itemize}
	\item document the code
	\item extract some specific data from the program (e.g., lint, metrics, and so on);
	\item automatically generate configuration files
	\item ...
\end{itemize}

\subsubsection{Meta-Data: A New Concept?}

Do annotations facility introduce a new concept in Java?

No, many applications require or produce files associated to a
specific class,

\begin{itemize}
	\item JAX-RPC, XML-RPC, SOAP, . . . ;
	\item EJB, JavaBean “BeanInfo” class
	\item ...
\end{itemize}

Creating and maintaining these associated files is painful.

Moreover many API need the use of special “placemarkers” 

Remote, Serializable, Cloneable, ... 

This feature adds a customizable mechanism for annotating classes, methods and fields

\subsubsection{Standard Annotations}

Standard annotation types are those provided "out-of-the-box"

\begin{itemize}
	\item \textbf{@Override}, to mark that a method overrides another method in its superclass
	\item \textbf{@Deprecated}, to indicate that the use of this method is discouraged; and
	\item \textbf{@SuppressWarning}, to turn off compiler warning for classes, methods, or field and variable initializers.
\end{itemize}

\begin{lstlisting}[language=Java]
@Override
public String toString() {
	return super.toString() + "[modified by subclass]" ;
}
\end{lstlisting}

\subsubsection{Categories of Custom Annotations}
There are three categories of custom annotations:
\begin{itemize}
	\item Marker Annotations, these are annotations without parameters or that uses default values for all parameters.
\begin{lstlisting}[language=Java]
@MarkerAnnotation
\end{lstlisting}	
	\item Single-Value Annotations, the annotations of this kind have just a single member named value
\begin{lstlisting}[language=Java]
@SingleValueAnnotation("some value");
\end{lstlisting}
	\item Full Annotations, these annotations exploits the full range of the annotation syntax
\begin{lstlisting}[language=Java]
@RevieVws({ // curly braces indicate an array of values
	@Review(grade=Review.Grade.EXCELLENT, reviewer="DF"),
	@Review(grade=Review.Grade.UNSATISFACTORY, reviewer="EG",
		comment="This method needs and @Override annotation.")
})
\end{lstlisting}
\end{itemize}

\subsubsection{Creating Custom Annotation Types}

Annotation types are, basically, Java interfaces
\begin{itemize}
	\item They look similar to a normal Java interface definition, but you use @interface which tells the compiler that you are writing an annotation type
\end{itemize}

Annotations’ members are defined by method signatures
\begin{itemize}
	\item these do not take any input parameters
	\item the return type defines the type of the member
\end{itemize}

\begin{lstlisting}[language=Java]
public @interface TODO {
	String value();
}
@TODO("Figure out the amount of interest per month")
	public void calculateInterest(float amount, float rate) {
	// need to finish this method later
}
\end{lstlisting}

Actually
\begin{itemize}
	\item when you write the method signature
	\item the compiler adds a member to the annotation with the same name and type after the method return type
	\item the method will be the selector for such a member
\end{itemize}

\subsubsection{Creating Custom Annotation Types (Cont'd)}

Similatly, we can define  annotation types with several members

\begin{lstlisting}[language=Java]
public @interface GroupTODO {
	public enum Severity {CRITICAL, IMPORTANT, TRIVIAL, DOCUMENTATION};
	Severity severity() default Severity.IMPORTANT;
	String item();
	String assignedTo();
}
@GroupTODO(
	severity=GroupTODO.Severity.CRITICAL,
	item="Figure out the amount of interest per month",
	assignedTo="Walter Cazzola"
)
public void calculateInterest(float amount, float rate) {
	// need to finish this method later
}
\end{lstlisting}

By omitting severity the default value would be used.


\subsubsection{Meta-Annotations, i.e., Annotations on Annotations}

The \textbf{meta-annotations} are annotations on annotations

There are four kind of predefined meta-annotations:

\begin{itemize}
	\item @Target, specifies which program elements (types, methods, ...) can have annotations of the defined type
	\item  @Retention, indicates whether an annotation is tossed out by the compiler or retained in the compiled class file.
	\item @Documented, indicates that the annotation should be considered as part of the public API of the annotated program element.
	\item @Inherited, is used on annotation types targeted at classes and indicates that the annotated type is an inherited one
\end{itemize}

\subsubsection{Reflecting on Annotations}

The easiest way to check for an annotation is by using the isAnnotationPresent() method.

\begin{lstlisting}[language=Java]
import java.lang.annotation.RetentionPolicy ;
import java.lang.annotation.Retention ;

@Retention(RetentionPolicy.RUNTIME)
public @interface TODO {
	String value();
}
\end{lstlisting}

\begin{lstlisting}[language=Java]
@TODO("Everything is still missing")
class Test {}
public class TestIsAnnotated {
	public static void main(String[] args) {
		Class c = Test.class;
		boolean todo = c.isAnnotationPresent(TODO.class);
		if (todo) System.out.println("The Test class is not completely implemented yet");
		else System.out.println("The Test class is completely implemented");
	}
}
\end{lstlisting}

\begin{lstlisting}[language=Java]
[23:01]cazzola@hymir:~/tsp>java TestIsAnnotated
The Test class is not completely implemented yet
\end{lstlisting}

\subsubsection{Reflecting on Annotations}

You can get the annotation’s members through its reification

\begin{lstlisting}[language=Java]
import java.lang.reflect.Method ;
class Test {
	@GroupTODO(
		severity=GroupTODO.Severity.CRITICAL,
		item="Figure out the amount of interest per month",
		assignedTo="Walter Cazzola")
	public void calculateInterest(float amount, float rate) {
		// need to finish this method later
	}
}
public class GetMembers {
	public static void main(String[] args) throws NoSuchMethodException {
	Class c = Test.class;
	Method element = c.getMethod("calculateInterest", float.class, float.class);
	GroupTODO groupTodo = element.getAnnotation(GroupTODO.class);
	String assignedTo = groupTodo.assignedTo();
	System.out.println("TODO item on Test is assigned to: '"+assignedTo+"'.");
	}
}
\end{lstlisting}